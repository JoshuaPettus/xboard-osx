% This is
%		diag.engl.tex
% Following is a short demonstration of the features of the elch-
% macros. This is not a complete documentation, but a commented
% plainTeX sources.
%
% Copyright (C) 1989-92  by Elmar Bartel.
%
% This program is free software; you can redistribute it and/or modify
% it under the terms of the GNU General Public License as published by
% the Free Software Foundation; either version 1, or (at your option)
% any later version.
%
% This program is distributed in the hope that it will be useful,
% but WITHOUT ANY WARRANTY; without even the implied warranty of
% MERCHANTABILITY or FITNESS FOR A PARTICULAR PURPOSE.  See the
% GNU General Public License for more details.
%
% You should have received a copy of the GNU General Public License
% along with this program; if not, write to the Free Software
% Foundation, Inc., 675 Mass Ave, Cambridge, MA 02139, USA.

%
\input elch
% Set the size of the board
\twelvechess
% Every board is enclosed by two commands:
\beginboard
\endboard
%
% These two instructions will produce a empty board
% without figures, author or stipulation.
%
% Between these two commands, you can put further commands,
% to specify authors, origin, which will be set above the
% board.
%
\beginboard
\author Erich Bartel;
\orig Jugendschach 45;
\endboard
% 
% Its important, that the argument to these commands are terminated
% with a semicolon. This semicolon has to be followed by a newline
% or blank character.
% There a further commands to specify more:
%
% \bnum Nummer;
%		It's possible to supply your own number for the
%		board, if you do not want consecutive numbering
%		of the boards.
%		If you want automatic numbering, but starting with
%		another number than 1, use \boardnumber=27 for example
%		to start with number 27.
%
% \dedic Widmung;
%		Here you can put you dedications, or other informations
%		This text appears below the origin and authors.
%
% All other commands place information below the board.
% There are stipulation, condition and remarks.
%
% \stip Forderung;
%		You put your stipulation here.
%
% \cond Bedingung;
%		If there are addtional conditions, put them here.
%		Should not be longer than the width of the board.
%
% All commands mentions up to now, accept more than one argument.
% They are seperated by comma: ', ' (comma blank).
% To specify more authors, you use for example:
% \author Erich Bartel, Elmar Bartel;
%
% When you need to put larger remarks or illings below the board,
% you can do it with the \rem command.
% Every argument (seperated by ', ' (comma blank) consists of two
% parts, seperated by '/' (slash). All remarks, will be lined up
% at the '/' which will not be printed.
%
% For example:
%
% \rem a)/Diagramm, b) wK\ra f4;
%
%
% to illustrate the use of these commands look at the following
% example, especially at the output produced.
%
\beginboard
\author Erich Bartel;
\orig Original;
\stip \#5;
\rem a)/Diagramm, bb)/wBa6 nach c7;
\endboard
%
% Now comes the heart of all this, the figures.
%
% Every figure get a uniqe abbreviation, consisting of the capital
% first character of the german name of the figure. (These have
% the advantage to be unique).
% 	B = Bauer,    Pawn
%	S = Springer, Knight
%	L = L"aufer,  Bishop
%	T = Turm,     Tower
%	D = Dame,     Queen
%	K = K"onig,   King
% These letter are preceded by the color-indicatiors:
%	w = weiss,    white
%	s = schwarz,  black
%	n = neutral,  neutral
% These resulting two-letter combinations may be followed by
% an orientation specification:
%       l = links,    left rotated by 90 degree
%	r = rechts,   right rotated by 90 degree
%       u = unten     turned upside down (180 degree)
% So for example a neutral king is named nK
% These figure specifications are followed by a list of field
% where this figure is to appear.
% White King on d4 look like this: wKd4
% Or a couple of black pawns: sBa3b6c7
% The secification of a whole position is a list of such figure
% specifications seperated by ', ' (comma blank) and terminated
% by '; ' (semicolon, blank or newline)
% The specification for the starting position looks like this:
%
\beginboard
\author Unkown;;
\orig Unkown;;
\pieces wBa2b2c2d2e2f2g2h2, wTa1h1, wSb1g1,
        wLc1f1, wDd1, wKe1, sBa7b7c7d7e7f7g7h7,
        sTa8h8, sSb8g8, sLc8f8, sDd8, sKe8;
\stip stipulation ?;
\cond No condition;
\rem a)/1.remark, b)/2.remark;
\endboard
\bigskip
%
% There are a few further commands:
% \gridtrue		The board will be printed with grid-lines
% \centeredtrue		The material above the board will be centered
% \figcntfalse		The figure count below will not be printed.
%
% Here come two brand new features:
%
% solutions:
% The solution of the problem is enclosed by  \beginsol \endsol.
% Everything put between these two commands will be gathered and
% later put out via the command \putsol
%
% The size of the board can be controlled by the commands
% 
% \twelvechess
% \elevenchess
% \tenchess
%
%
\bye
